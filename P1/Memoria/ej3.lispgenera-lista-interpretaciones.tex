\begin{aibox}{\function}
\begin{alltt}
;; genera-lista-interpretaciones

SYNTAX: genera-lista-interpretaciones (lst) 
RECIBE   : Lista de N símbolos atómicos
EVALÚA A : Lista de 2^N posibles interpretaciones
             (lista de pares (<símbolo> <valor de verdad>))

\end{alltt}
\end{aibox}

\begin{aibox}{\examples}
\begin{alltt}
(genera-lista-interpretaciones '(P I L))
;;; (((P T) (I T) (L T)) ((P T) (I T) (L NIL)) ((P T) (I NIL) (L T))
 ((P T) (I NIL) (L NIL)) ((P NIL) (I T) (L T)) ((P NIL) (I T) (L NIL))
 ((P NIL) (I NIL) (L T)) ((P NIL) (I NIL) (L NIL)))
 (genera-lista-interpretaciones '(A))
;;; (((A T) (A NIL)))
\end{alltt}

\end{aibox}

\begin{aibox}{\comments}
\paragraph{}
Primero generamos todas las posibles interpretaciones de cada átomo (que solo puede ser T y NIL) y formamos una lista de listas de la forma (((Atomo1 T) (Atomo1 NIL)) ((Atomo2 T)(Atomo2 NIL)) ...)
\paragraph{}
Ahora solo falta combinar estas listas (con la función del ejericico 2.3) para conseguir todas las posibles combinaciones.
\paragraph{}
Al realizar este ejercicio nos dimos cuenta de un error en la función combine-lst-of-lst que tuvimos que corregir.
\end{aibox}

\begin{aibox}{\pseudocode}
\begin{alltt}
genera-lista-interpretaciones (lst)
    para cada elemento en lst
        añadir ( combinar(elemento, (T, NIL), aux)
    combine-lst-of-lst (aux)
\end{alltt}
\end{aibox}
\begin{aibox}{\code}

\begin{alltt}
(defun genera-lista-interpretaciones (lst) 
    (combine-lst-of-lst (mapcar #'(lambda (x) (combine-elt-lst x '(T NIL))) lst)))
\end{alltt}
\end{aibox}
