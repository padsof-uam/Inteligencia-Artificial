
\begin{aibox}{\function}
;; combine-elt-lst

SYNTAX: combine-elt-lst (elt lst)
\end{aibox}

\begin{aibox}{\examples}

\begin{alltt}
(combine-elt-lst 'a '(1 2 3)); ((A 1) (A 2) (A 3))

(combine-elt-lst 'a '()); NIL
\end{alltt}

\end{aibox}

\begin{aibox}{\comments}
En una primera implementación utilizando \emph{cons} nos dimos cuenta de que no era del todo correcto, ya que el rest del último elemento no era NIL por defecto. En cambio si utilizamos \emph{list} esto queda bien definido.
\end{aibox}
\begin{aibox}{\pseudocode}
El pseudocódigo que pensamos para resolver el problema:

\begin{alltt}
combine-elt-lst (elemento,lista)
    Para cada atomo en lista
        añadir ((elemento,atomo),resultado)
\end{alltt}
\end{aibox}
\begin{aibox}{\code}

\begin{alltt}

(defun combine-elt-lst (elt lst)
            (mapcar #'(lambda (x) (list elt x)) lst))

\end{alltt}
\end{aibox}
