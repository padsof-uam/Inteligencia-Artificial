\begin{aibox}{\function}
;; h-recursive

SYNTAX: h-recursive (x w sigma) 
\end{aibox}

\begin{aibox}{\examples}
\begin{alltt}
(H-RECURSIVE '(0.1 -0.5 0.7) '(-0.1 0.2 0.3) #'logit); 0.4750208
(H-RECURSIVE '(0.1 -0.5 0.7) '(-0.1 0.2 0.3) #'TANHIP); 0.09966801

(H-RECURSIVE '(0.1 -0.5 0.7) '(-0.1 1) #'TANHIP); ERROR
(H-RECURSIVE '(0.1 -0.5 0.7) '() #'tanhip) ERROR
\end{alltt}
\end{aibox}

\begin{aibox}{\comments}
Salen los mismos resultados que con la función H-MAPCAR.
\end{aibox}
\begin{aibox}{\answers}

\end{aibox}
\begin{aibox}{\othercomments}
Utilizamos la llamada a la función auxiliar para optimizar el uso de la recursión. Con el uso de un acumulador conseguimos evitar la cadena de devoluciones para resolver la recursión.

Incluimos el código de la función auxiliar empleada en el apartado de código.
\end{aibox}
\begin{aibox}{\pseudocode}

\end{aibox}
\begin{aibox}{\code}

\begin{alltt}

(defun h-recursive (x w sigma) 
        (funcall sigma (h-recursive-aux x w 0)))

(defun h-recursive-aux (x w acc)
    (if (null x)
          acc
    (h-recursive-aux (rest x) (rest w) ( + (* (first x)  (first w)) acc))))


\end{alltt}
\end{aibox}
