\begin{aibox}{\function}
\begin{alltt}
;; extrae-simbolos

SYNTAX: extrae-simbolos (expr)
RECIBE : fórmula bien formada en cualquier formato (expr)
EVAlÚA A: Lista de símbolos atómicos (sin repeticiones)
	utilizados en la fórmula bien formada. El orden en la lista no es
	relevante.

\end{alltt}
\end{aibox}

\begin{aibox}{\examples}
\begin{alltt}
(extrae-simbolos '((=> (^ P I) L) (=> (¬ P) (¬ L)) (¬ P) (L)))
;; (I P L)
(extrae-simbolos '((<=> (¬ (P ^ L)) (v (¬ (=> K P)) K P))))
;; (L K P)
\end{alltt}

\end{aibox}

\begin{aibox}{\comments}
\paragraph{}
Para extraer los símbolos, aplanamos la lista (nos quitamos los paréntesis anidados), eliminamos todo lo que no sea un símbolo (\texttt{symbol-p} devuelve T si el argumento no es ni un conector ni \texttt{T} o \texttt{nil}) y eliminamos posibles duplicados.
\end{aibox}

\begin{aibox}{\pseudocode}
\begin{alltt}
symbols = ()

for atom in (flatten expr):
    if (symbol-p atom)
        append atom to symbols

remove duplicates in symbols
return symbols

\end{alltt}
\end{aibox}

\begin{aibox}{\code}

\begin{alltt}

(defun extrae-simbolos (expr)
    (remove-duplicates (remove-if-not #'symbol-p (flatten expr))))

\end{alltt}
\end{aibox}
