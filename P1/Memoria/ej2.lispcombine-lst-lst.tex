\begin{aibox}{\function}
;; combine-lst-lst

SYNTAX: combine-lst-lst (lst1 lst2)
\end{aibox}

\begin{aibox}{\examples}
\begin{alltt}
(combine-lst-lst '(a b c) '(1 2))
; ((A 1) (A 2) (B 1) (B 2) (C 1) (C 2))

(combine-lst-lst '(a b c) '()); NIL

(combine-lst-lst '() '(a b c)); NIL
\end{alltt}

\end{aibox}

\begin{aibox}{\comments}
Utilizamos la función definida en el ejercicio anterior (combine-elt-lst) lo que facilita la tarea.
\end{aibox}
\begin{aibox}{\answers}
No hay preguntas en el enunciado.
\end{aibox}
\begin{aibox}{\othercomments}
En este ejercicio vimos la diferencia entre mapcan y mapcar. Necesitábamos que el resultado fuera 1 única lista con todos los elementos y no una lista de listas. Este conocimiento fue realmente útil en la resolución del siguiente ejercicio.
\end{aibox}
\begin{aibox}{\pseudocode}
Esta primera aproximación (traducida a programación funcional utilizando \emph{mapcar}) luego fue modificada (por el uso de mapcan en vez de mapcar) para que el resultado fuera una única lista de átomos.
\begin{alltt}
combine-lst-lst (lst1,lst2)
    Para cada eltemento en lst1:
        combine-elt-lst (elemento,lst2)
\end{alltt}
\end{aibox}
\begin{aibox}{\code}

\begin{alltt}

(defun combine-lst-lst (lst1 lst2)
            (mapcan #'(lambda (x) (combine-elt-lst x lst2)) lst1))

\end{alltt}
\end{aibox}
