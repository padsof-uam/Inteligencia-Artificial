\begin{aibox}{\function}
\begin{alltt}
;; encuentra-modelos

SYNTAX: encuentra-modelos (kb)
RECIBE   : base de conocimiento (KB)
EVALÚA A : lista de interpretaciones que son modelo para KB

\end{alltt}
\end{aibox}

\begin{aibox}{\examples}
\begin{alltt}
(encuentra-modelos '((=> A (¬ H)) (<=> P (^ A  H)) (=> H P))) 
; (((A T) (P NIL) (H NIL)) ((A NIL) (P NIL) (H NIL)))

(encuentra-modelos '((=> (^ P I)  L)  (=> (¬ P) (¬ L)) (¬ P) L)) 
; NIL

\end{alltt}

\end{aibox}

\begin{aibox}{\comments}
Esta función simplemente devuelve una lista con las interpretaciones posibles de la base de conocimiento que pasan el filtro dado por la función anónima que determina si la interpretación es modelo. Usamos \texttt{remove-if-not}, que es una función no destructiva.
\end{aibox}
\begin{aibox}{\pseudocode}
\begin{alltt}
ints = interpretaciones(simbolos de kb)
eliminar int de ints si no es interpretacion-modelo-p de kb

\end{alltt}
\end{aibox}
\begin{aibox}{\code}

\begin{alltt}

(defun encuentra-modelos (kb) 
  (remove-if-not 
    #'(lambda (int) (interpretacion-modelo-p kb int)) 
    (genera-lista-interpretaciones (extrae-simbolos kb))))

\end{alltt}
\end{aibox}
