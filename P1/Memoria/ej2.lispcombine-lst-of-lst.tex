\begin{aibox}{\function}
\begin{alltt}
;; combine-lst-of-lst

SYNTAX: combine-lst-of-lst (lst)
RECIBE: una lista de listas
EVALÚA A: una lista de todas las posibles combinaciones de los elementos del argumento. 
\end{alltt}
\end{aibox}

\begin{aibox}{\examples}
\begin{alltt}
 (combine-lst-of-lst '((a b c) (+ -) (1 2 3)))

;((A + 1) (A + 2) (A + 3) (A - 1) (A - 2) (A - 3) (B + 1) (B + 2) (B + 3)
; (B - 1) (B - 2) (B - 3) (C + 1) (C + 2) (C + 3) (C - 1) (C - 2) (C - 3))

(combine-lst-of-lst '((a b) (+ -) (1 2) (NIL T)))

; ((A + 1 NIL) (A + 1 T) (A + 2 NIL) (A + 2 T) (A - 1 NIL) (A - 1 T) (A - 2 NIL)
; (A - 2 T) (B + 1 NIL) (B + 1 T) (B + 2 NIL) (B + 2 T) (B - 1 NIL) (B - 1 T)
; (B - 2 NIL) (B - 2 T))

(combine-lst-of-lst '((a b) (+ -) (1 2) ()))
; NIL

(combine-lst-of-lst '(((A t) (A nil)))) ; ((A T) (A NIL))
\end{alltt}

\end{aibox}

\begin{aibox}{\comments}

Esta función utiliza una combinación de elemento con lista distinta. 
\paragraph{}
La hemos definido de forma recursiva (que nos parecía más fácil de pensar) pero utilizando una combinación de elemento con lista distinta a la del ejercicio 1, pero sólo en el caso recursivo. En el caso base hemos mantenido la función del ejercicio 2.1. 
\paragraph{}
Esta decisión la tomamos para obtener como resultado una única lista de listas, con únicamente 2 niveles y no más. Empleando la función del ejercicio 2.1 no funcionaba correctamente este requisito.
\paragraph{}
Las funciones auxiliares definidas son combine-lst-lst-app que simplemente llama a la función combine-elt-lst-app que utiliza \emph{cons} en vez de \emph{list}.
\paragraph{}
Hemos pensado que no hacía falta rellenar la ficha de las funciones auxiliares (debido a su gran similitud con las soluciones de los ejercicios 2.1 y 2.2).
\paragraph{}
La primera comprobación (no incluida en un primer momento) la tuvimos en cuenta al realizar el tercer ejercicio. En caso de recibir una lista de listas de esta forma: '(((A t) (A nil))) devolvía NIL.
\end{aibox}
\newpage
\begin{aibox}{\pseudocode}
\begin{alltt}
funcion CombinarListaDeListas:
    Si (la lista tiene 2 listas)
        CombinarListas(sublista1,sublista2)
    sino
        CombinarListas (Primera\_Sublista, CombinarListaDeListas (RestoDeSublistas))
\end{alltt}
\end{aibox}

\begin{aibox}{\code}

\begin{alltt}

(defun combine-lst-of-lst (lst)
    (if (null (cdr lst))
        (car lst)
    (if (null (cddr lst))
            (combine-lst-lst (car lst) (cadr lst))
    (combine-lst-lst-app (car lst) (combine-lst-of-lst (cdr lst))))))

\end{alltt}
Las funciones auxiliares son:
\begin{alltt} 
(defun combine-elt-lst-app (elt lst)
     (mapcar #'(lambda (x) (cons elt x)) lst))

(defun combine-lst-lst-app (lst1 lst2)
     (mapcan #'(lambda (x) (combine-elt-lst-app x lst2)) lst1))
\end{alltt}
\end{aibox}
