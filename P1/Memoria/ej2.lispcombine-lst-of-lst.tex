\begin{aibox}{\function}
;; combine-lst-of-lst

SYNTAX: combine-lst-of-lst (lst)
\end{aibox}

\begin{aibox}{\examples}
\begin{alltt}
 (combine-lst-of-lst '((a b c) (+ -) (1 2 3)))

;((A + 1) (A + 2) (A + 3) (A - 1) (A - 2) (A - 3) (B + 1) (B + 2) (B + 3)
; (B - 1) (B - 2) (B - 3) (C + 1) (C + 2) (C + 3) (C - 1) (C - 2) (C - 3))

(combine-lst-of-lst '((a b) (+ -) (1 2) (NIL T)))

; ((A + 1 NIL) (A + 1 T) (A + 2 NIL) (A + 2 T) (A - 1 NIL) (A - 1 T) (A - 2 NIL)
; (A - 2 T) (B + 1 NIL) (B + 1 T) (B + 2 NIL) (B + 2 T) (B - 1 NIL) (B - 1 T)
; (B - 2 NIL) (B - 2 T))

(combine-lst-of-lst '((a b) (+ -) (1 2) ()))
; NIL
\end{alltt}

\end{aibox}

\begin{aibox}{\comments}
Esta función utiliza una combinación de elemento con lista distinta. 

La hemos definido de forma recursiva (que nos parecía más fácil de pensar) pero utilizando una combinación de elemento con lista distinta a la del ejercicio 1, pero sólo en el caso recursivo. En el caso base hemos mantenido la función del ejercicio 2.1. 

Esta decisión la tomamos para obtener como resultado una única lista de listas, con únicamente 2 niveles y no más. Empleando la función del ejercicio 2.1 no funcionaba correctamente este requisito.

Las funciones auxiliares definidas son combine-lst-lst-app que simplemente llama a la función combine-elt-lst-app que utiliza \emph{cons} en vez de \emph{list}.

Hemos pensado que no hacía falta rellenar la ficha de las funciones auxiliares (debido a su gran similitud con las soluciones de los ejercicios 2.1 y 2.2).

\end{aibox}
\begin{aibox}{\pseudocode}
\begin{alltt}
funcion CombinarListaDeListas:
	Si (la lista tiene 2 listas)
		CombinarListas(sublista1,sublista2)
	sino
		CombinarListas (Primera\_Sublista, CombinarListaDeListas (RestoDeSublistas))
\end{alltt}
\end{aibox}
\begin{aibox}{\code}

\begin{alltt}
(defun combine-lst-of-lst (lst)
    (if (null (cddr lst))
            (combine-lst-lst (car lst) (cadr lst))
    (combine-lst-lst-app (car lst) (combine-lst-of-lst (cdr lst)))))

\end{alltt}
\end{aibox}
