\begin{aibox}{\function}
\begin{alltt}
;; eval-fbf

SYNTAX: eval-fbf (fbf int)
RECIBE: fórmula bien formada e interpretación
EVALÚA A : Valor de verdad de la fbf con la interpretación dada

\end{alltt}
\end{aibox}

\begin{aibox}{\examples}
\begin{alltt}
(eval-fbf '(<=> T NIL) '())
; NIL
(eval-fbf '(=> A NIL) '((A T)))
; NIL
(eval-fbf '(<=> P (^ A  H)) '((A NIL) (P NIL) (H T)))
; T
(eval-fbf '(<=> (v A P H) (^ A P H)) '((A NIL) (P NIL) (H T)))
; NIL
\end{alltt}

\end{aibox}

\begin{aibox}{\comments}
Incluimos esta función en una ficha aparte (aunque no responda a un apartado en concreto) porque nos parece importante documentarla.

La función \texttt{evaluator} devuelve la función correspondiente al operador pasado como argumento. Después de evaluar todos los elementos de la lista de forma recursiva con \texttt{eval-fbf}, aplicamos el evaluador con los elementos de la lista como argumentos.
\end{aibox}

\begin{aibox}{\pseudocode}
\begin{alltt}
si fbf es un átomo
	obtener valor de verdad de fbf con interp.
si no
	sym = primer elemento de fbf
	evaluator = obtener función evaluadora para el operador
	results = eval-fbf(resto de elementos de la fbf)
	llamar a evaluator con results como argumentos

\end{alltt}
\end{aibox}
\begin{aibox}{\code}

\begin{alltt}
(defun eval-fbf (fbf int)
   	(if (atom fbf)
         (getint fbf int)
         (apply 
           (evaluator (first fbf)) 
           (mapcar #'(lambda (f) (eval-fbf f int)) (rest fbf)))))

\end{alltt}
\end{aibox}
