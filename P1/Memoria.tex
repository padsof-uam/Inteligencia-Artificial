\documentclass{aitemplate}

\usepackage{textcomp}
\usepackage{ai}
\usepackage{alltt}
\usepackage{tikz}

\begin{document}

\printtitle{1}

\section*{Ejercicio 1}
\begin{aibox}{\function}
;; h-mapcar

SYNTAX: h-mapcar (x w sigma) 
\end{aibox}

\begin{aibox}{\examples}
\begin{alltt}
(h-mapcar '(0.1 -0.5 0.7) '() #'tanhip); 0.0
(h-mapcar '(0.1 -0.5 0.7) '(1 2) #'tanhip); -0.7162978
(h-mapcar '(0.1 -0.5 0.7) '(-0.1 0.2 0.3) #'tanhip); 0.09966801
(h-mapcar '(0.1 -0.5 0.7) '(-0.1 0.2 0.3) #'logit); 0.4750208
\end{alltt}
\end{aibox}

\begin{aibox}{\comments}

\end{aibox}
\begin{aibox}{\answers}
No hay preguntas.
\end{aibox}

\begin{aibox}{\othercomments}

\end{aibox}
\begin{aibox}{\pseudocode}
\end{aibox}
\begin{aibox}{\code}

\begin{alltt}
;;%% code

(defun h-mapcar (x w sigma) 
    (funcall sigma 
        (reduce #'+
            (mapcar #'* x w))))

\end{alltt}
\end{aibox}

%\newpage
%\begin{aibox}{\function}
;; h-recursive-aux

SYNTAX: h-recursive-aux (x w acc)
\end{aibox}

\begin{aibox}{\examples}
\begin{alltt}
\end{alltt}

\end{aibox}

\begin{aibox}{\comments}

\end{aibox}
\begin{aibox}{\answers}

\end{aibox}
\begin{aibox}{\othercomments}

\end{aibox}
\begin{aibox}{\pseudocode}

\end{aibox}
\begin{aibox}{\code}

\begin{alltt}
;;%% code

(defun h-recursive-aux (x w acc)
        (if (null x)
            acc
        (h-recursive-aux (rest x) (rest w) ( + (* (first x)  (first w)) acc))))
\end{alltt}
\end{aibox}

\newpage
\begin{aibox}{\function}
\begin{alltt}
;; h-recursive

SYNTAX: h-recursive (x w sigma) 
RECIBE: dos listas (\textbf{x} y \textbf{w}) y una función sigma
EVALÚA A: el valor de la función sigma aplicada al producto escalar de las 2 listas.
\end{alltt}
\end{aibox}

\begin{aibox}{\examples}
\begin{alltt}
(H-RECURSIVE '(0.1 -0.5 0.7) '(-0.1 0.2 0.3) #'logit); 0.4750208
(H-RECURSIVE '(0.1 -0.5 0.7) '(-0.1 0.2 0.3) #'TANHIP); 0.09966801

(H-RECURSIVE '(0.1 -0.5 0.7) '(-0.1 1) #'TANHIP); ERROR
(H-RECURSIVE '(0.1 -0.5 0.7) '() #'tanhip) ERROR
\end{alltt}
\end{aibox}

\begin{aibox}{\comments}
Salen los mismos resultados que con la función H-MAPCAR, como podíamos esperar. 

Simplemente hay que tener cuidado con que las listas deben tener la misma longitud para que funcione correctamente.

Utilizamos la llamada a la función auxiliar para optimizar el uso de la recursión. Con el uso de un acumulador conseguimos evitar la cadena de devoluciones para resolver la recursión.

Incluimos el código de la función auxiliar empleada en el apartado de código.
\end{aibox}

\begin{aibox}{\pseudocode}
Esta es la versión inicial en la que pensamos que no incluía el acumulador: 
\begin{alltt}
h-recursiva (x w sigma)
     (sigma(first(x)*first(w)) + h-recursive(rest(x),rest(w),sigma))
\end{alltt}
\end{aibox}
\begin{aibox}{\code}

\begin{alltt}

(defun h-recursive (x w sigma) 
        (funcall sigma (h-recursive-aux x w 0)))

\end{alltt}
Y la función auxiliar es:
\begin{alltt}
(defun h-recursive-aux (x w acc)
    (if (null x)
          acc
    (h-recursive-aux (rest x) (rest w) ( + (* (first x)  (first w)) acc))))


\end{alltt}
\end{aibox}

%\newpage
%\input{Memoria/ej1.lisplogit.tex}
%\newpage
%\input{Memoria/ej1.lisptanhip.tex}
%\newpage
%\begin{aibox}{\function}
;; combine-elt-lst-app

SYNTAX: combine-elt-lst-app (elt lst)
\end{aibox}

\begin{aibox}{\examples}
\begin{alltt}
\end{alltt}

\end{aibox}

\begin{aibox}{\comments}

\end{aibox}
\begin{aibox}{\answers}

\end{aibox}
\begin{aibox}{\othercomments}

\end{aibox}
\begin{aibox}{\pseudocode}

\end{aibox}
\begin{aibox}{\code}

\begin{alltt}
;;%% code

(defun combine-elt-lst-app (elt lst)
            (mapcar #'(lambda (x) (append (list elt) x)) lst))

\end{alltt}
\end{aibox}

\newpage

\section*{Ejercicio 2}
\begin{aibox}{\function}
;; combine-elt-lst

SYNTAX: combine-elt-lst (elt lst)
\end{aibox}

\begin{aibox}{\examples}

\begin{alltt}
(combine-elt-lst 'a '(1 2 3)); ((A 1) (A 2) (A 3))

(combine-elt-lst 'a '()); NIL
\end{alltt}

\end{aibox}

\begin{aibox}{\comments}
En una primera implementación utilizando \emph{cons} nos dimos cuenta de que no era del todo correcto, ya que el rest del último elemento no era NIL por defecto. En cambio si utilizamos \emph{list} esto queda bien definido.
\end{aibox}
\begin{aibox}{\answers}
No hay preguntas en el enunciado.
\end{aibox}
\begin{aibox}{\othercomments}

\end{aibox}
\begin{aibox}{\pseudocode}
El pseudocódigo que pensamos para resolver el problema:

\begin{alltt}
combine-elt-lst (elemento,lista)
    Para cada atomo en lista
        añadir ((elemento,atomo),resultado)
\end{alltt}
\end{aibox}
\begin{aibox}{\code}

\begin{alltt}

(defun combine-elt-lst (elt lst)
            (mapcar #'(lambda (x) (list elt x)) lst))

\end{alltt}
\end{aibox}

%\newpage
%\begin{aibox}{\function}
;; combine-lst-lst-app

SYNTAX: combine-lst-lst-app (lst1 lst2)
\end{aibox}

\begin{aibox}{\examples}
\begin{alltt}
\end{alltt}

\end{aibox}

\begin{aibox}{\comments}

\end{aibox}
\begin{aibox}{\answers}

\end{aibox}
\begin{aibox}{\othercomments}

\end{aibox}
\begin{aibox}{\pseudocode}

\end{aibox}
\begin{aibox}{\code}

\begin{alltt}
;;%% code

(defun combine-lst-lst-app (lst1 lst2)
            (mapcan #'(lambda (x) (combine-elt-lst-app x lst2)) lst1))

\end{alltt}
\end{aibox}

\newpage
\begin{aibox}{\function}
;; combine-lst-lst

SYNTAX: combine-lst-lst (lst1 lst2)
\end{aibox}

\begin{aibox}{\examples}
\begin{alltt}
(combine-lst-lst '(a b c) '(1 2))
; ((A 1) (A 2) (B 1) (B 2) (C 1) (C 2))

(combine-lst-lst '(a b c) '()); NIL

(combine-lst-lst '() '(a b c)); NIL
\end{alltt}

\end{aibox}

\begin{aibox}{\comments}
Utilizamos la función definida en el ejercicio anterior (combine-elt-lst) lo que facilita la tarea.
\end{aibox}
\begin{aibox}{\answers}
No hay preguntas en el enunciado.
\end{aibox}
\begin{aibox}{\othercomments}
En este ejercicio vimos la diferencia entre mapcan y mapcar. Necesitábamos que el resultado fuera 1 única lista con todos los elementos y no una lista de listas. Este conocimiento fue realmente útil en la resolución del siguiente ejercicio.
\end{aibox}
\begin{aibox}{\pseudocode}
Esta primera aproximación (traducida a programación funcional utilizando \emph{mapcar}) luego fue modificada (por el uso de mapcan en vez de mapcar) para que el resultado fuera una única lista de átomos.
\begin{alltt}
combine-lst-lst (lst1,lst2)
    Para cada eltemento en lst1:
        combine-elt-lst (elemento,lst2)
\end{alltt}
\end{aibox}
\begin{aibox}{\code}

\begin{alltt}

(defun combine-lst-lst (lst1 lst2)
            (mapcan #'(lambda (x) (combine-elt-lst x lst2)) lst1))

\end{alltt}
\end{aibox}

\newpage
\begin{aibox}{\function}
;; combine-lst-of-lst

SYNTAX: combine-lst-of-lst (lst)
\end{aibox}

\begin{aibox}{\examples}
\begin{alltt}
 (combine-lst-of-lst '((a b c) (+ -) (1 2 3)))

;((A + 1) (A + 2) (A + 3) (A - 1) (A - 2) (A - 3) (B + 1) (B + 2) (B + 3)
; (B - 1) (B - 2) (B - 3) (C + 1) (C + 2) (C + 3) (C - 1) (C - 2) (C - 3))

(combine-lst-of-lst '((a b) (+ -) (1 2) (NIL T)))

; ((A + 1 NIL) (A + 1 T) (A + 2 NIL) (A + 2 T) (A - 1 NIL) (A - 1 T) (A - 2 NIL)
; (A - 2 T) (B + 1 NIL) (B + 1 T) (B + 2 NIL) (B + 2 T) (B - 1 NIL) (B - 1 T)
; (B - 2 NIL) (B - 2 T))

(combine-lst-of-lst '((a b) (+ -) (1 2) ()))
; NIL

(combine-lst-of-lst '(((A t) (A nil)))) ; ((A T) (A NIL))
\end{alltt}

\end{aibox}

\begin{aibox}{\comments}
Esta función utiliza una combinación de elemento con lista distinta. 

La hemos definido de forma recursiva (que nos parecía más fácil de pensar) pero utilizando una combinación de elemento con lista distinta a la del ejercicio 1, pero sólo en el caso recursivo. En el caso base hemos mantenido la función del ejercicio 2.1. 

Esta decisión la tomamos para obtener como resultado una única lista de listas, con únicamente 2 niveles y no más. Empleando la función del ejercicio 2.1 no funcionaba correctamente este requisito.

Las funciones auxiliares definidas son combine-lst-lst-app que simplemente llama a la función combine-elt-lst-app que utiliza \emph{cons} en vez de \emph{list}.

Hemos pensado que no hacía falta rellenar la ficha de las funciones auxiliares (debido a su gran similitud con las soluciones de los ejercicios 2.1 y 2.2).

La primera comprobación (no incluida en un primer momento) la tuvimos en cuenta al realizar el tercer ejercicio. En caso de recibir una lista de listas de esta forma: '(((A t) (A nil))) devolvía NIL.

\end{aibox}
\begin{aibox}{\pseudocode}
\begin{alltt}
funcion CombinarListaDeListas:
	Si (la lista tiene 2 listas)
		CombinarListas(sublista1,sublista2)
	sino
		CombinarListas (Primera\_Sublista, CombinarListaDeListas (RestoDeSublistas))
\end{alltt}
\end{aibox}
\begin{aibox}{\code}

\begin{alltt}

(defun combine-lst-of-lst (lst)
	(if (null (cdr lst))
		(car lst)
	(if (null (cddr lst))
			(combine-lst-lst (car lst) (cadr lst))
	(combine-lst-lst-app (car lst) (combine-lst-of-lst (cdr lst))))))


\end{alltt}
\end{aibox}

\newpage

\section*{Ejercicio 3}
\begin{aibox}{\function}
\begin{alltt}
;; eval-fbf

SYNTAX: eval-fbf (fbf int)
RECIBE: fórmula bien formada e interpretación
EVALÚA A : Valor de verdad de la fbf con la interpretación dada

\end{alltt}
\end{aibox}

\begin{aibox}{\examples}
\begin{alltt}
(eval-fbf '(<=> T NIL) '())
; NIL
(eval-fbf '(=> A NIL) '((A T)))
; NIL
(eval-fbf '(<=> P (^ A  H)) '((A NIL) (P NIL) (H T)))
; T
(eval-fbf '(<=> (v A P H) (^ A P H)) '((A NIL) (P NIL) (H T)))
; NIL
\end{alltt}

\end{aibox}

\begin{aibox}{\comments}
Incluimos esta función en una ficha aparte (aunque no responda a un apartado en concreto) porque nos parece importante documentarla.

La función \texttt{evaluator} devuelve la función correspondiente al operador pasado como argumento. Después de evaluar todos los elementos de la lista de forma recursiva con \texttt{eval-fbf}, aplicamos el evaluador con los elementos de la lista como argumentos.
\end{aibox}

\begin{aibox}{\pseudocode}
\begin{alltt}
si fbf es un átomo
	obtener valor de verdad de fbf con interp.
si no
	sym = primer elemento de fbf
	evaluator = obtener función evaluadora para el operador
	results = eval-fbf(resto de elementos de la fbf)
	llamar a evaluator con results como argumentos

\end{alltt}
\end{aibox}
\begin{aibox}{\code}

\begin{alltt}
(defun eval-fbf (fbf int)
   	(if (atom fbf)
         (getint fbf int)
         (apply 
           (evaluator (first fbf)) 
           (mapcar #'(lambda (f) (eval-fbf f int)) (rest fbf)))))

\end{alltt}
\end{aibox}

\newpage
\begin{aibox}{\function}
;; encuentra-modelos

SYNTAX: encuentra-modelos (kb) 
\end{aibox}

\begin{aibox}{\examples}
\begin{alltt}
(encuentra-modelos '((=> A (¬ H)) (<=> P (^ A  H)) (=> H P))) ;
 (((A T) (P NIL) (H NIL)) ((A NIL) (P NIL) (H NIL)))

(encuentra-modelos '((=> (^ P I)  L)  (=> (¬ P) (¬ L)) (¬ P) L)) ;
 NIL

\end{alltt}

\end{aibox}

\begin{aibox}{\comments}

\end{aibox}
\begin{aibox}{\answers}

\end{aibox}
\begin{aibox}{\othercomments}

\end{aibox}
\begin{aibox}{\pseudocode}

\end{aibox}
\begin{aibox}{\code}

\begin{alltt}

(defun encuentra-modelos (kb) 
  (remove-if-not 
    #'(lambda (int) (interpretacion-modelo-p kb int)) 
    (genera-lista-interpretaciones (extrae-simbolos kb))))

\end{alltt}
\end{aibox}

\newpage
\begin{aibox}{\function}
\begin{alltt}
;; extrae-simbolos

SYNTAX: extrae-simbolos (expr)
RECIBE : fórmula bien formada en cualquier formato (expr)
EVAlÚA A: Lista de símbolos atómicos (sin repeticiones)
	utilizados en la fórmula bien formada. El orden en la lista no es
	relevante.

\end{alltt}
\end{aibox}

\begin{aibox}{\examples}
\begin{alltt}
(extrae-simbolos '((=> (^ P I) L) (=> (¬ P) (¬ L)) (¬ P) (L)))
;; (I P L)
(extrae-simbolos '((<=> (¬ (P ^ L)) (v (¬ (=> K P)) K P))))
;; (L K P)
\end{alltt}

\end{aibox}

\begin{aibox}{\comments}
Para extraer los símbolos, aplanamos la lista (nos quitamos los paréntesis anidados), eliminamos todo lo que no sea un símbolo (\texttt{symbol-p} devuelve T si el argumento no es ni un conector ni \texttt{T} o \texttt{nil}) y eliminamos posibles duplicados.
\end{aibox}

\begin{aibox}{\pseudocode}
\begin{alltt}
symbols = ()

for atom in (flatten expr):
	if (symbol-p atom)
		append atom to symbols

remove duplicates in symbols
return symbols

\end{alltt}
\end{aibox}

\begin{aibox}{\code}

\begin{alltt}

(defun extrae-simbolos (expr)
	(remove-duplicates (remove-if-not #'symbol-p (flatten expr))))

\end{alltt}
\end{aibox}

\newpage
\begin{aibox}{\function}
\begin{alltt}
;; genera-lista-interpretaciones

SYNTAX: genera-lista-interpretaciones (lst) 
RECIBE   : Lista de N símbolos atómicos
EVALÚA A : Lista de 2^N posibles interpretaciones
             (lista de pares (<símbolo> <valor de verdad>))

\end{alltt}
\end{aibox}

\begin{aibox}{\examples}
\begin{alltt}
(genera-lista-interpretaciones '(P I L))
;;; (((P T) (I T) (L T)) ((P T) (I T) (L NIL)) ((P T) (I NIL) (L T))
 ((P T) (I NIL) (L NIL)) ((P NIL) (I T) (L T)) ((P NIL) (I T) (L NIL))
 ((P NIL) (I NIL) (L T)) ((P NIL) (I NIL) (L NIL)))
 (genera-lista-interpretaciones '(A))
;;; (((A T) (A NIL)))
\end{alltt}

\end{aibox}

\begin{aibox}{\comments}
Primero generamos todas las posibles interpretaciones de cada átomo (que solo puede ser T y NIL) y formamos una lista de listas de la forma (((Atomo1 T) (Atomo1 NIL)) ((Atomo2 T)(Atomo2 NIL)) ...)

Ahora solo falta combinar estas listas (con la función del ejericico 2.3) para conseguir todas las posibles combinaciones.

Al realizar este ejercicio nos dimos cuenta de un error en la función combine-lst-of-lst que tuvimos que corregir.
\end{aibox}

\begin{aibox}{\pseudocode}
\begin{alltt}
genera-lista-interpretaciones (lst)
    para cada elemento en lst
        añadir ( combinar(elemento, (T, NIL), aux)
    combine-lst-of-lst (aux)
\end{alltt}
\end{aibox}
\begin{aibox}{\code}

\begin{alltt}
(defun genera-lista-interpretaciones (lst) 
    (combine-lst-of-lst (mapcar #'(lambda (x) (combine-elt-lst x '(T NIL))) lst)))
\end{alltt}
\end{aibox}

\newpage
\begin{aibox}{\function}
\begin{alltt}
;; interpretacion-modelo-p

SYNTAX: interpretacion-modelo-p (kb interpretacion) 
RECIBE: - kb. Lista de fbfs.
		- interpretacion. Lista con asignaciones de valores de verdad a cada símbolo.

\end{alltt}
\end{aibox}

\begin{aibox}{\examples}
\begin{alltt}
(interpretacion-modelo-p 
 '((<=> A (¬ H)) (<=> P (^ A  H)) (<=> H P)) '((A NIL) (P NIL) (H T)))
 ;;; NIL

(interpretacion-modelo-p
 '((<=> A (¬ H)) (<=> P (^ A  H)) (<=> H P)) '((A T) (P NIL) (H NIL)))
;;; T
\end{alltt}
\end{aibox}

\begin{aibox}{\comments}
Evaluamos todas las fbf de la base de conocimiento y nos aseguramos que todas sean \texttt{T} con \texttt{every identity}.
\end{aibox}

\begin{aibox}{\pseudocode}
\begin{alltt}
evaluations = eval-fbf for each fbf in kb
return every evaluation in evaluations is T

\end{alltt}
\end{aibox}
\begin{aibox}{\code}

\begin{alltt}

(defun interpretacion-modelo-p (kb interpretacion) 
  (every #'identity (mapcar #'(lambda (fbf) (eval-fbf fbf interpretacion)) kb)))

\end{alltt}
\end{aibox}

\newpage
\begin{aibox}{\function}
;; SAT-p

SYNTAX: SAT-p (kb)
\end{aibox}

\begin{aibox}{\examples}
\begin{alltt}
(SAT-p '((<=> A (¬ H)) (<=> P (^ A H)) (<=> H P))) ; T

(SAT-p '((=> (^ P I) L) (=> (¬ P) (¬ L)) (¬ P) L)) ; NIL
\end{alltt}

\end{aibox}

\begin{aibox}{\comments}

\end{aibox}
\begin{aibox}{\answers}

\end{aibox}
\begin{aibox}{\othercomments}

\end{aibox}
\begin{aibox}{\pseudocode}

\end{aibox}
\begin{aibox}{\code}

\begin{alltt}

(defun SAT-p (kb)
  (not (null (encuentra-modelos kb))))

\end{alltt}
\end{aibox}

\newpage
%\input{Memoria/ej3.lisptruth-value-p.tex}
%\newpage

\section*{Ejercicio 4}

\paragraph*{I)} El algoritmo de búsqueda en anchura expande un nodo, y después explora todos sus hijos. La implementación se haría con una cola FIFO: exploramos un nodo y añadimos sus hijos a la cola de abiertos, de tal forma que el orden de exploración va de forma descendiente, de nodos más cercanos a la raíz a los más alejados.

\


\end{document}
