\documentclass[nochap]{apuntes}

\usepackage{listings}
\author{Guillermo Julián y Víctor de Juan}
\date{10-Abril}
\title{Inteligencia Artificial - Practica 4}

\lstset{
	 inputencoding=utf8
}

\begin{document}
\maketitle
\section*{Parte 1}
\paragraph{Realiza todas las pruebas posibles para conocer a fondo el funcionamiento del motor de inferencia. ¿Que ocurre si introducimos variables en nuestra hipotesis, como '(pertenece ?x (2 5 3 6 7))?}

\begin{lstlisting}
(set-hypothesis-list '((pertenece ?x (2 5 3 6 Y))))
(motor-inferencia)
;; res-> (((?X . 2)) ((?X . 5)) ((?X . 3)) ((?X . 6)) ((?X . Y)))
\end{lstlisting}


\paragraph{ Haz trazas del codigo siguiendo las funciones más importantes\\}

\begin{verbatim}
3. Trace: (CONSULTA '((PERTENECE 1 (2 5 1 6 7))))
4. Trace: (FIND-HYPOTHESIS-VALUE '(PERTENECE 1 (2 5 1 6 7)))
5. Trace: 
(EVAL-RULE '(PERTENECE 1 (2 5 1 6 7))
 '(R2 (PERTENECE ?E.34078 (?_ . ?XS.34079)) :-
   ((PERTENECE ?E.34078 ?XS.34079))))
6. Trace: (CONSULTA '((PERTENECE 1 (5 1 6 7))))
7. Trace: (FIND-HYPOTHESIS-VALUE '(PERTENECE 1 (5 1 6 7)))
8. Trace: 
(EVAL-RULE '(PERTENECE 1 (5 1 6 7))
 '(R2 (PERTENECE ?E.34102 (?_ . ?XS.34103)) :-
   ((PERTENECE ?E.34102 ?XS.34103))))
9. Trace: (CONSULTA '((PERTENECE 1 (1 6 7))))
10. Trace: (FIND-HYPOTHESIS-VALUE '(PERTENECE 1 (1 6 7)))
11. Trace: (EVAL-RULE '(PERTENECE 1 (1 6 7)) '(R1 (PERTENECE ?E.34125 (?E.34125 . ?_))))
11. Trace: EVAL-RULE ==> (((NIL)))
11. Trace: 
(EVAL-RULE '(PERTENECE 1 (1 6 7))
 '(R2 (PERTENECE ?E.34126 (?_ . ?XS.34127)) :-
   ((PERTENECE ?E.34126 ?XS.34127))))
12. Trace: (CONSULTA '((PERTENECE 1 (6 7))))
13. Trace: (FIND-HYPOTHESIS-VALUE '(PERTENECE 1 (6 7)))
14. Trace: 
(EVAL-RULE '(PERTENECE 1 (6 7))
 '(R2 (PERTENECE ?E.34150 (?_ . ?XS.34151)) :-
   ((PERTENECE ?E.34150 ?XS.34151))))
15. Trace: (CONSULTA '((PERTENECE 1 (7))))
16. Trace: (FIND-HYPOTHESIS-VALUE '(PERTENECE 1 (7)))
17. Trace: 
(EVAL-RULE '(PERTENECE 1 (7))
 '(R2 (PERTENECE ?E.34174 (?_ . ?XS.34175)) :-
   ((PERTENECE ?E.34174 ?XS.34175))))
18. Trace: (CONSULTA '((PERTENECE 1 NIL)))
19. Trace: (FIND-HYPOTHESIS-VALUE '(PERTENECE 1 NIL))
19. Trace: FIND-HYPOTHESIS-VALUE ==> NIL
18. Trace: CONSULTA ==> NIL
17. Trace: EVAL-RULE ==> NIL
16. Trace: FIND-HYPOTHESIS-VALUE ==> NIL
15. Trace: CONSULTA ==> NIL
14. Trace: EVAL-RULE ==> NIL
13. Trace: FIND-HYPOTHESIS-VALUE ==> NIL
12. Trace: CONSULTA ==> NIL
11. Trace: EVAL-RULE ==> NIL
10. Trace: FIND-HYPOTHESIS-VALUE ==> (((NIL)))
10. Trace: (CONSULTA 'NIL)
10. Trace: CONSULTA ==> (((NIL)))
9. Trace: CONSULTA ==> (((NIL)))
8. Trace: EVAL-RULE ==> (((NIL)))
7. Trace: FIND-HYPOTHESIS-VALUE ==> (((NIL)))
7. Trace: (CONSULTA 'NIL)
7. Trace: CONSULTA ==> (((NIL)))
6. Trace: CONSULTA ==> (((NIL)))
5. Trace: EVAL-RULE ==> (((NIL)))
4. Trace: FIND-HYPOTHESIS-VALUE ==> (((NIL)))
4. Trace: (CONSULTA 'NIL)
4. Trace: CONSULTA ==> (((NIL)))
3. Trace: CONSULTA ==> (((NIL)))
\end{verbatim}


\paragraph{ Explica mediante un esquema de llamadas (utilizando un  ́árbol a por ejemplo) qué pasos va haciendo el motor de inferencia hasta llegar a la solución. ¿En qué se parece esto a Prolog? ¿En qué se diferencia?\\}

GUILEEEEEEEEEEEEEEEEEEEEEEEEEEEEEEEEEEEEEEEEEEEEEEEEEE


\section{Predicados implementados}

\subsection*{Encolar}
\begin{lstlisting}
;; Caso base: encolamos 1 elemento a una lista vacía.
(set-hypothesis-list '((encolar () 1 ?L)))
(motor-inferencia)
;;res -> (((?L 1)))

;; Caso recursivo: encolamos 1 elemento a una lista no vacía.
(set-hypothesis-list '((encolar (1 2 3) 4 ?Rs)))
(motor-inferencia)
;;res -> (((?RS 1 2 3 4)))

;; Caso base: encolamos 1 elemento a una lista vacía.
(set-hypothesis-list '((encolar () 1 (1))))
(motor-inferencia)
;;res -> (((NIL)))
\end{lstlisting}

\subsection*{Concatenar}
\begin{lstlisting}

;; Caso base: agregamos una lista de un elemento a una lista vacía.
(set-hypothesis-list '((concatenar () (5) ?Xs)))
(motor-inferencia)
;;res -> (((?XS 5)))

;; Caso base: agregamos una lista de 1 elemento a una lista no vacía (como encolar).
(set-hypothesis-list '((concatenar (6 7) (5) ?Xs)))
(motor-inferencia)
;;res -> (((?XS 6 7 5)))

;; Caso recursivo: 2 listas no vacías
(set-hypothesis-list '((concatenar () (1 2) ?Xs)))
(motor-inferencia)
;;res -> (((?XS 1 2)))
\end{lstlisting}


\subsection*{Invertir}
\begin{lstlisting}
(set-hypothesis-list '((invertir () ?Zs)))
(motor-inferencia)
;;res -> (((?ZS)))

(set-hypothesis-list '((invertir (1 2 3 4 5) ?X)))
(motor-inferencia)
;;res -> (5 4 3 2 1)


;;; Estos ejemplos se comen la pila.
;(set-hypothesis-list '((invertir  ?X () )))
;(motor-inferencia)
;; Program stack overflow.
;(set-hypothesis-list '((invertir  ?X (1 2 3 4 5))))
;(motor-inferencia)
;; Program stack overflow.

\end{lstlisting}


\section*{Parte 2}


\subsection*{Productorio}
\begin{lstlisting}
(set-hypothesis-list '((productorio () ?R)))
(motor-inferencia)
;;res-> (((?R . 1)))

(set-hypothesis-list '((productorio (10) ?R)))
(motor-inferencia)
;;res-> (((?R . 10)))

(set-hypothesis-list '((productorio (5 10) ?R)))
(motor-inferencia)
;;res-> (((?R . 10)))

(set-hypothesis-list '((productorio (1 2 3 4) ?R)))
(motor-inferencia)
;;res-> (((?R . 24)))
\end{lstlisting}

\subsection*{Elemento N de una lista}

\begin{lstlisting}

(set-hypothesis-list '((posicionN 0 (A B C D E) ?X)))
(motor-inferencia)
;;res-> NIL

(set-hypothesis-list '((posicionN 7 (A B C D E) ?X)))
(motor-inferencia)
;;res-> NIL

(set-hypothesis-list '((posicionN 3 () ?X)))
(motor-inferencia)
;;res-> NIL

(set-hypothesis-list '((posicionN 4 (A B C D E) ?X)))
(motor-inferencia)
;;res-> (((?X . D)))

(set-hypothesis-list '((posicionN -2 (4 5) ?X)))
(motor-inferencia)
;;res-> NIL
\end{lstlisting}


\end{document}