\documentclass[nochap]{apuntes}

\usepackage{listings}
\lstset{language=Lisp}
\author{Guillermo Julián y Víctor de Juan}
\date{10-Abril}
\title{Inteligencia Artificial - Practica 4}


\begin{document}
\maketitle
\section*{Parte 1}

\paragraph{Realiza todas las pruebas posibles para conocer a fondo el funcionamiento del motor de inferencia. ¿Que ocurre si introducimos variables en nuestra hipotesis, como '(pertenece ?x (2 5 3 6 ?Y))?\\\\}

Ocurre lo mismo que si fueran constantes. Esta regla también unifica si \textit{X=Y}, con lo que en el resultado observaremos una unificación como ((?X . ?Y)).\\

\begin{lstlisting}[frame=single]

(set-hypothesis-list '((pertenece ?x (2 5 3 6 Y))))
(motor-inferencia)
;; (((?X . 2)) ((?X . 5)) ((?X . 3)) ((?X . 6)) ((?X . ?Y)))
\end{lstlisting}

\paragraph{Haz las trazas del códifo siguiendo las funciones más importantes}

\begin{verbatim}
1. Trace: (CONSULTA '((PERTENECE 1 (2 5 1 6 7))))
 2. Trace: (FIND-HYPOTHESIS-VALUE '(PERTENECE 1 (2 5 1 6 7)))
  3. Trace: (EVAL-RULE '(PERTENECE 1 (2 5 1 6 7))
    '(R2 (PERTENECE ?E.120564 (?_ . ?XS.120565)) :-
     ((PERTENECE ?E.120564 ?XS.120565))))
    4. Trace: (CONSULTA '((PERTENECE 1 (5 1 6 7))))
     5. Trace: (FIND-HYPOTHESIS-VALUE '(PERTENECE 1 (5 1 6 7)))
      6. Trace: (EVAL-RULE '(PERTENECE 1 (5 1 6 7))
        '(R2 (PERTENECE ?E.120612 (?_ . ?XS.120613)) :-
         ((PERTENECE ?E.120612 ?XS.120613))))
         7. Trace: (CONSULTA '((PERTENECE 1 (1 6 7))))
          8. Trace: (FIND-HYPOTHESIS-VALUE '(PERTENECE 1 (1 6 7)))
           9. Trace: (EVAL-RULE '(PERTENECE 1 (1 6 7)) 
               '(R1 (PERTENECE ?E.120659 (?E.120659 . ?_))))
           9. Trace: EVAL-RULE ==> (((NIL)))
           9. Trace: (EVAL-RULE '(PERTENECE 1 (1 6 7))
                   '(R2 (PERTENECE ?E.120660 (?_ . ?XS.120661)) :-
                    ((PERTENECE ?E.120660 ?XS.120661))))
            10. Trace: (CONSULTA '((PERTENECE 1 (6 7))))
             11. Trace: (FIND-HYPOTHESIS-VALUE '(PERTENECE 1 (6 7)))
              12. Trace: (EVAL-RULE '(PERTENECE 1 (6 7))
               '(R2 (PERTENECE ?E.120708 (?_ . ?XS.120709)) :-
               ((PERTENECE ?E.120708 ?XS.120709))))
               13. Trace: (CONSULTA '((PERTENECE 1 (7))))
                14. Trace: (FIND-HYPOTHESIS-VALUE '(PERTENECE 1 (7)))
                 15. Trace: (EVAL-RULE '(PERTENECE 1 (7))
                     '(R2 (PERTENECE ?E.120756 (?_ . ?XS.120757)) :-
                      ((PERTENECE ?E.120756 ?XS.120757))))
                  16. Trace: (CONSULTA '((PERTENECE 1 NIL)))
                   17. Trace: (FIND-HYPOTHESIS-VALUE '(PERTENECE 1 NIL))
                   17. Trace: FIND-HYPOTHESIS-VALUE ==> NIL
                  16. Trace: CONSULTA ==> NIL
                 15. Trace: EVAL-RULE ==> NIL
                14. Trace: FIND-HYPOTHESIS-VALUE ==> NIL
               13. Trace: CONSULTA ==> NIL
              12. Trace: EVAL-RULE ==> NIL
             11. Trace: FIND-HYPOTHESIS-VALUE ==> NIL
            10. Trace: CONSULTA ==> NIL
            9. Trace: EVAL-RULE ==> NIL
           8. Trace: FIND-HYPOTHESIS-VALUE ==> (((NIL)))
           8. Trace: (CONSULTA 'NIL)
           8. Trace: CONSULTA ==> (((NIL)))
          7. Trace: CONSULTA ==> (((NIL)))
         6. Trace: EVAL-RULE ==> (((NIL)))
        5. Trace: FIND-HYPOTHESIS-VALUE ==> (((NIL)))
        5. Trace: (CONSULTA 'NIL)
        5. Trace: CONSULTA ==> (((NIL)))
       4. Trace: CONSULTA ==> (((NIL)))
     3. Trace: EVAL-RULE ==> (((NIL)))
    2. Trace: FIND-HYPOTHESIS-VALUE ==> (((NIL)))
    2. Trace: (CONSULTA 'NIL)
   2. Trace: CONSULTA ==> (((NIL)))
  1. Trace: CONSULTA ==> (((NIL)))
(((NIL)))
\end{verbatim}
\paragraph{Explica mediante un esquema de llamadas (utilizando un árbol por ejemplo) qué pasos va haciendo el motor de inferencia hasta llegar a la solución. ¿En qué se parece esto a Prolog? ¿En qué se diferencia?\\\\}

Cuando encontramos una regla con la que unificar la parte izquierda, tomamos como objeto de estudio cada una de las partes de la derecha (las implicaciones necesarias) y empezamos a buscar reglas que unifiquen con esa parte izquierda y mediante un proceso recursivo, (como indican las trazas) llegamos a un árbol binario en este caso (porque el predicadod \textit{pertenece} sólo tiene 2 reglas).

Está indicado si unifica o no, y en caso de unificar, la llamada recursiva con la parte de la derecha de la regla (que es la que implica la parte izquierda).

Como encontramos una coincidencia, devolvemos esa unificación.

En este caso se han omitido las llamadas que realiza este motor de inferencia a las funciones concretas (como consulta, find-hypothesis-value y eval-rule) ya que esa información se encuentra en la traza y deja el diagrama en árbol más sencillo, claro y comprensible. El árbol simplemente modeliza el proceso de \textit{backtracking} seguido en el caso \textit{(set-hypothesis-list '((pertenece 1 (2 5 1 6 7))))}
\newpage
\easyimgw{arb.png}{Arbol de backtracking}{Arb}{1.05}

En cuanto a parecidos, ambos funcionan con \textit{back-tracking} (encadenamiento hacia atrás), como comprobamos en el árbol de llamadas.

Hay 3 grandes diferencias (aparte de la sintaxis obviamente) que son:
\begin{itemize}
\item Prolog no devuelve elementos repetidos  en una salida, mientras que el motor de inferencia si las permite, es decir:
\begin{lstlisting}[frame=single]

(set-hypothesis-list '((pertenece ?x (2 2 2 2))))
(motor-inferencia)
;; res -> (((?X . 2)) ((?X . 2)) ((?X . 2)) ((?X . 2)))
\end{lstlisting}
Mientras que Prolog devolvería simplemente \textit{X = 2}, ya que ese es el único valor con el que unifica.
\item En Prolog las salidas booleanas son \textit{True} o \textit{False}. En el motor de inferencia el \textit{False} se representa con \textit{Nil} y el \textit{true} se representa con \textit{(((NIL)))}.
\item El motor de inferencia no para hasta que no ha encontrado todas las posibles soluciones. Prolog permite parar cuando haya encontrado una.
\end{itemize}

\section{Predicados implementados}

\subsection*{Encolar}
Las reglas implementadas son:\\
\begin{lstlisting}[frame=single]

(R3 (encolar () ?E  (?E)))
(R4 (encolar (?Y . ?L) ?E (?Y . ?Z)) :- ((encolar ?L ?E ?Z)))
\end{lstlisting}

Y los ejemplos que hemos probado han sido:\\

\begin{lstlisting}[frame=single]

;; Caso base: encolamos 1 elemento a una lista vacía.
(set-hypothesis-list '((encolar () 1 ?L)))
(motor-inferencia)
;;res -> (((?L 1)))

;; Caso recursivo: encolamos 1 elemento a una lista no vacía.
(set-hypothesis-list '((encolar (1 2 3) 4 ?Rs)))
(motor-inferencia)
;;res -> (((?RS 1 2 3 4)))

;; Caso base: encolamos 1 elemento a una lista vacía.
(set-hypothesis-list '((encolar () 1 (1))))
(motor-inferencia)
;;res -> (((NIL)))
\end{lstlisting}

\subsection*{Concatenar}
Las reglas implementadas son:\\
\begin{lstlisting}[frame=single]

(R7 (concatenar () ?L ?L))
(R8 (concatenar (?X . ?R) ?L (?X . ?R2)) 
	:- ((concatenar ?R ?L ?R2)))
\end{lstlisting}
Y los ejemplos que hemos probado han sido:\\

\begin{lstlisting}[frame=single]


;; Caso base: agregamos una lista de 
;;		un elemento a una lista vacía.
(set-hypothesis-list '((concatenar () (5) ?Xs)))
(motor-inferencia)
;;res -> (((?XS 5)))

;; Caso base: agregamos una lista de 1 elemento
;;		 a una lista no vacía (como encolar).
(set-hypothesis-list '((concatenar (6 7) (5) ?Xs)))
(motor-inferencia)
;;res -> (((?XS 6 7 5)))

;; Caso recursivo: 2 listas no vacías
(set-hypothesis-list '((concatenar () (1 2) ?Xs)))
(motor-inferencia)
;;res -> (((?XS 1 2)))
\end{lstlisting}


\subsection*{Invertir}
Las reglas implementadas son:\\
\begin{lstlisting}[frame=single]

(R9 (invertir () ()))
(R10 (invertir (?X . ?Y) ?Z) 
		:- ((invertir ?Y ?H) (encolar ?H ?X ?Z)))
\end{lstlisting}
Y los ejemplos que hemos probado han sido:\\

\begin{lstlisting}[frame=single]


(set-hypothesis-list '((invertir () ?Zs)))
(motor-inferencia)
;;res -> (((?ZS)))

(set-hypothesis-list '((invertir (1 2 3 4 5) ?X)))
(motor-inferencia)
;;res -> (5 4 3 2 1)


;;; Estos ejemplos se comen la pila, ya que no está pensado
;;;	 el motor de inferencia para recibir
;;;	 variables donde deberían ser constantes.

;(set-hypothesis-list '((invertir  ?X () )))
;(motor-inferencia)
;; Program stack overflow.
;(set-hypothesis-list '((invertir  ?X (1 2 3 4 5))))
;(motor-inferencia)
;; Program stack overflow.

\end{lstlisting}


\section*{Parte 2}


\subsection*{Productorio}
Las reglas implementadas son:\\
\begin{lstlisting}[frame=single]

(R14 (productorio () 1))
(R15 (productorio (?X . ?R) ?Z) :- 
	((productorio ?R ?P) (?= ?Z (?eval (* ?P ?X)))))
\end{lstlisting}
Y los ejemplos que hemos probado han sido:\\

\begin{lstlisting}[frame=single]

(set-hypothesis-list '((productorio () ?R)))
(motor-inferencia)
;;res-> (((?R . 1)))

(set-hypothesis-list '((productorio (10) ?R)))
(motor-inferencia)
;;res-> (((?R . 10)))

(set-hypothesis-list '((productorio (5 10) ?R)))
(motor-inferencia)
;;res-> (((?R . 10)))

(set-hypothesis-list '((productorio (1 2 3 4) ?R)))
(motor-inferencia)
;;res-> (((?R . 24)))
\end{lstlisting}

\subsection*{Elemento N de una lista}
Las reglas implementadas son:\\
\begin{lstlisting}[frame=single]

(R16 (posicionN 1 (?X . ?_) ?X))
(R17 (posicionN ?X (?_ . ?L) ?R) 
		:- ((?= T (?eval (> ?X 1)))
			(?= ?P (?eval (- ?X 1)))
			(posicionN ?P ?L ?R)))
\end{lstlisting}
Y los ejemplos que hemos probado han sido:\\


\begin{lstlisting}[frame=single]


(set-hypothesis-list '((posicionN 0 (A B C D E) ?X)))
(motor-inferencia)
;;res-> NIL

(set-hypothesis-list '((posicionN 7 (A B C D E) ?X)))
(motor-inferencia)
;;res-> NIL

(set-hypothesis-list '((posicionN 3 () ?X)))
(motor-inferencia)
;;res-> NIL

(set-hypothesis-list '((posicionN 4 (A B C D E) ?X)))
(motor-inferencia)
;;res-> (((?X . D)))

(set-hypothesis-list '((posicionN -2 (4 5) ?X)))
(motor-inferencia)
;;res-> NIL
\end{lstlisting}

\subsection*{Map}
Las reglas implementadas son:\\
\begin{lstlisting}[frame=single]

(R18 (doblar ?X ?R) :- ((?= ?R (?eval (* ?X 2)))))
(R19 (map ?_ () ()))
(R20 (map ?Pred (?First . ?Rest) ?Retval) :- (
			 (?Pred ?First ?FirstRes)
			 (concatenar (?FirstRes) ?Recur ?Retval)
			 (map ?Pred ?Rest ?Recur)
			 ))
\end{lstlisting}
Y los ejemplos que hemos probado han sido:\\


\begin{lstlisting}[frame=single]

(set-hypothesis-list '((map doblar () ?Rs)))
(motor-inferencia)
;;res-> (((?RS)))

(set-hypothesis-list '((map doblar (1) ?Rs)))
(motor-inferencia)
;;res-> (((?RS 2)))

(set-hypothesis-list '((map doblar (2 1) ?Rs)))
(motor-inferencia)
;;res-> (((?RS 4 2)))

(set-hypothesis-list '((map doblar (2 1 3 0 25) ?Rs)))
(motor-inferencia)
;;res-> (((?RS 4 2 6 0 50)))
\end{lstlisting}

\subsection*{Reduce}
Las reglas implementadas son:\\
\begin{lstlisting}[frame=single]

(R21 (neutro + 0))
	(R22 (neutro * 1))
	(R23 (+ ?A ?B (?eval (+ ?A ?B))))
	(R24 (* ?A ?B (?eval (* ?A ?B))))
	(R25 (reduce ?Pred () ?R) :- ((neutro ?Pred ?R)))
	(R26 (reduce ?Pred (?First . ?Rest) ?Retval) :- (
		(reduce ?Pred ?Rest ?Partial)
		(?Pred ?First ?Partial ?Retval)
		))
\end{lstlisting}
Y los ejemplos que hemos probado han sido:\\


\begin{lstlisting}[frame=single]


(set-hypothesis-list '((+ 1 2 ?R)))
(motor-inferencia)
;; res -> (((?R . 3)))

(set-hypothesis-list '((neutro + ?R)))
(motor-inferencia)
;; res -> (((?R . 0)))

(set-hypothesis-list '((reduce + (1 2 3) ?R)))
(motor-inferencia)
;; res -> (((?RS . 6)))

(set-hypothesis-list '((reduce + () ?Rs)))
(motor-inferencia)
;; res -> (((?RS . 0)))

(set-hypothesis-list '((neutro * ?R)))
(motor-inferencia)
;; res -> (((?R . 1)))

(set-hypothesis-list '((reduce * (1 2 3 4) ?R)))
(motor-inferencia)
;; res -> (((?RS . 24)))

(set-hypothesis-list '((reduce * () ?Rs)))
(motor-inferencia)
;; res -> (((?RS . 1))

(set-hypothesis-list '((reduce * (0) ?Rs)))
(motor-inferencia)
;; res -> (((?RS . 0))

(set-hypothesis-list '((reduce + (1) ?Rs)))
(motor-inferencia)
;; res -> (((?RS . 1))
\end{lstlisting}

\end{document}