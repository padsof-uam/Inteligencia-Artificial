\begin{aibox}{\function}
\begin{alltt}
insert-nodes (nodes lst-nodes strategy)


IN:     nodes: lista de nodos para insertar.
       lst-nodes:    lista de nodos en la que insertar.
       strategy:    estrategia que seguir a la hora de insertar.
OUT:    una lista de nodos ordenados de acuerdo a la estrategia.

\end{alltt}
\end{aibox}

\begin{aibox}{\examples}
\begin{alltt}
Examples
(insert-nodes 
    (list *node-00* *node-01* *node-02*) 
    (insert-nodes 
        (list *node-00* *node-01* *node-02*) 
        *lst-nodes-0*
        *uniform-cost*)
    *uniform-cost*)

\#S(NODE :STATE PROSERPINA 
 :PARENT NIL
 :ACTION NIL
 :DEPTH 12 :G 10 :H NIL :F 20)
\#S(NODE :STATE AVALON 
 :PARENT NIL
 :ACTION NIL
 :DEPTH 0 :G 0 :H NIL :F 0)
\#S(NODE :STATE MALLORY
 :PARENT \#S(NODE :STATE PROSERPINA :PARENT NIL :ACTION NIL :DEPTH 12 :G 10 :H NIL :F 20)
 :ACTION \#S(ACTION :NAME NAVIGATE-WORM-HOLE :ORIGIN PROSERPINA :FINAL MALLORY :COST 6)
 :DEPTH 13 :G 16 :H 7 :F 23)
\#S(NODE :STATE SIRTIS
 :PARENT \#S(NODE :STATE PROSERPINA :PARENT NIL :ACTION NIL :DEPTH 12 :G 10 :H NIL :F 20)
 :ACTION \#S(ACTION :NAME NAVIGATE-WORM-HOLE :ORIGIN PROSERPINA :FINAL SIRTIS :COST 7)
 :DEPTH 13 :G 17 :H 0 :F 17)
\#S(NODE :STATE KENTARES
 :PARENT \#S(NODE :STATE PROSERPINA :PARENT NIL :ACTION NIL :DEPTH 12 :G 10 :H NIL :F 20)
 :ACTION \#S(ACTION :NAME NAVIGATE-WORM-HOLE :ORIGIN PROSERPINA :FINAL KENTARES :COST 1)
 :DEPTH 13 :G 11 :H 4 :F 15)
\#S(NODE :STATE MALLORY
 :PARENT \#S(NODE :STATE PROSERPINA :PARENT NIL :ACTION NIL :DEPTH 12 :G 10 :H NIL :F 20)
 :ACTION \#S(ACTION :NAME NAVIGATE-WHITE-HOLE :ORIGIN PROSERPINA :FINAL MALLORY :COST 7)
 :DEPTH 13 :G 17 :H 7 :F 24)
\#S(NODE :STATE AVALON
 :PARENT \#S(NODE :STATE PROSERPINA :PARENT NIL :ACTION NIL :DEPTH 12 :G 10 :H NIL :F 20)
 :ACTION \#S(ACTION :NAME NAVIGATE-WHITE-HOLE :ORIGIN PROSERPINA :FINAL AVALON :COST 2)
 :DEPTH 13 :G 12 :H 5 :F 17)
\#S(NODE :STATE DAVION
 :PARENT \#S(NODE :STATE PROSERPINA :PARENT NIL :ACTION NIL :DEPTH 12 :G 10 :H NIL :F 20)
 :ACTION \#S(ACTION :NAME NAVIGATE-WHITE-HOLE :ORIGIN PROSERPINA :FINAL DAVION :COST 4)
 :DEPTH 13 :G 14 :H 1 :F 15)
\#S(NODE :STATE SIRTIS
 :PARENT \#S(NODE :STATE PROSERPINA :PARENT NIL :ACTION NIL :DEPTH 12 :G 10 :H NIL :F 20)
 :ACTION \#S(ACTION :NAME NAVIGATE-WHITE-HOLE :ORIGIN PROSERPINA :FINAL SIRTIS :COST 10)
 :DEPTH 13 :G 20 :H 0 :F 20)
\#S(NODE :STATE KENTARES :PARENT NIL :ACTION NIL :DEPTH 2 :G 50 :H NIL :F 50))



\end{alltt}
\end{aibox}

\begin{aibox}{\comments}
Inserta una lista de nodos en otra (ya ordenada) de acuerdo con una estrategia.

\end{aibox}

\begin{aibox}{\pseudocode}
\begin{alltt}
   insert-nodes(nodes lst strategy)
       iterar at1 en nodes
           iterar at2 en lst
               si ato1 < at2
                   insertar ato1 y desplazar el resto
Opción b (que encontramos mejor,
porque modificar una lista sobre la que estamos iterando es peligroso):
   insert-nodes(nodes lst strategy acc)
       iterar at1 en nodes
           iterar at2 en lst
               si ato1 < at2
                   insertar ato1 en acc
                   insert-nodes 
                         (rest(nodes) lst-nodes strategy acc)
               sino
                   insertar ato2 en acc
                   insertar-nodes (nodes rest(lst-nodes) strategy acc)

\end{alltt}
\end{aibox}

\begin{aibox}{\code}
\begin{alltt}
(defun \_aux-insert-nodes (nodes lst-nodes acc strategy)
    (if (null nodes)
        (append acc lst-nodes)
        (if (null lst-nodes)
            (append acc nodes)
            (if  (funcall (strategy-node-compare-p strategy) (car nodes) (car lst-nodes))

                (\_aux-insert-nodes 
                    (cdr nodes)
                    lst-nodes
                    (append acc (list (car nodes)))
                    strategy)
                (\_aux-insert-nodes
                    nodes
                    (cdr lst-nodes)
                    (append acc (list (car lst-nodes)))
                    strategy)))))

(defun insert-nodes (nodes lst-nodes strategy)
    (\_aux-insert-nodes nodes lst-nodes '() strategy))




\end{alltt}
\end{aibox}
