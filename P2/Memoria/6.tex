\begin{aibox}{\function}
\begin{alltt}
expand-node (nodeArg problem)


IN:     node: el nodo a expandir
       problem: estructura con toda la información necesaria
OUT: lista de los nodos hijos

\end{alltt}
\end{aibox}

\begin{aibox}{\examples}
\begin{alltt}
Examples
(setf *lst-nodes-0*
    (expand-node *node-00* *galaxy-M35*))

(\#S(NODE :STATE MALLORY
 :PARENT  \#S(NODE :STATE PROSERPINA :PARENT NIL :ACTION NIL :DEPTH 12 :G 10 :H NIL :F 20)
 :ACTION  \#S(ACTION :NAME NAVIGATE-WORM-HOLE :ORIGIN PROSERPINA :FINAL MALLORY :COST 6)
 :DEPTH 13 :G 16 :H 7 :F 23)
\#S(NODE :STATE SIRTIS
 :PARENT  \#S(NODE :STATE PROSERPINA :PARENT NIL :ACTION NIL :DEPTH 12 :G 10 :H NIL :F 20)
 :ACTION  \#S(ACTION :NAME NAVIGATE-WORM-HOLE :ORIGIN PROSERPINA :FINAL SIRTIS :COST 7)
 :DEPTH 13 :G 17 :H 0 :F 17)
\#S(NODE :STATE KENTARES
 :PARENT  \#S(NODE :STATE PROSERPINA :PARENT NIL :ACTION NIL :DEPTH 12 :G 10 :H NIL :F 20)
 :ACTION  \#S(ACTION :NAME NAVIGATE-WORM-HOLE :ORIGIN PROSERPINA :FINAL KENTARES :COST 1)
 :DEPTH 13 :G 11 :H 4 :F 15)
\#S(NODE :STATE MALLORY
 :PARENT  \#S(NODE :STATE PROSERPINA :PARENT NIL :ACTION NIL :DEPTH 12 :G 10 :H NIL :F 20)
 :ACTION  \#S(ACTION :NAME NAVIGATE-WHITE-HOLE :ORIGIN PROSERPINA :FINAL MALLORY :COST 7)
 :DEPTH 13 :G 17 :H 7 :F 24)
\#S(NODE :STATE AVALON
 :PARENT  \#S(NODE :STATE PROSERPINA :PARENT NIL :ACTION NIL :DEPTH 12 :G 10 :H NIL :F 20)
 :ACTION  \#S(ACTION :NAME NAVIGATE-WHITE-HOLE :ORIGIN PROSERPINA :FINAL AVALON :COST 2)
 :DEPTH 13 :G 12 :H 5 :F 17)
\#S(NODE :STATE DAVION
 :PARENT  \#S(NODE :STATE PROSERPINA :PARENT NIL :ACTION NIL :DEPTH 12 :G 10 :H NIL :F 20)
 :ACTION  \#S(ACTION :NAME NAVIGATE-WHITE-HOLE :ORIGIN PROSERPINA :FINAL DAVION :COST 4)
 :DEPTH 13 :G 14 :H 1 :F 15)
\#S(NODE :STATE SIRTIS
 :PARENT  \#S(NODE :STATE PROSERPINA :PARENT NIL :ACTION NIL :DEPTH 12 :G 10 :H NIL :F 20)
 :ACTION  \#S(ACTION :NAME NAVIGATE-WHITE-HOLE :ORIGIN PROSERPINA :FINAL SIRTIS :COST 10)
 :DEPTH 13 :G 20 :H 0 :F 20))



\end{alltt}
\end{aibox}

\begin{aibox}{\comments}
Expande el nodo dado.

Para ello buscaremos en las estructuras de problem 
la información sobre a qué planetas podemos viajar (qué nodos son los sucesores)

y crearemos una estructura nodo para cada sucesor con toda la información necesaria.  No comprobamos si el nodo es solución (que es lo primero antes de expandir nodo).

Dejamos esa comprobación para la tarea superior.

\end{aibox}

\begin{aibox}{\pseudocode}
\begin{alltt}
Iterar atomo en planetas\_destino
    make-nodo (node,atomo)

\end{alltt}
\end{aibox}

\begin{aibox}{\code}
\begin{alltt}
(mapcan \#'(lambda(x)  
        (fncall x (node-state *node-00*))) 
        (problem-operators *galaxy-M35*))

(defun expand-node (nodeArg problem)
    (let ((lst
            (mapcan
                \#'(lambda(x)  
                    (fncall x (node-state nodeArg))) 
                (problem-operators problem))))
            (mapcar \#'(lambda(x) 
                (let (
                    (g (+ (action-cost x) (node-g nodeArg)))
                    (h (fncall 
                            (problem-fn-h problem)
                            (action-final x))))
                        (make-node
                            :state (action-final x)
                            :parent nodeArg
                            :action x
                            :depth (+ 1 (node-depth nodeArg))
                            :g g
                            :h h
                            :f (+ g h)))) lst)))






\end{alltt}
\end{aibox}
