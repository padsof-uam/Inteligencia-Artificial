\begin{aibox}{\function}
\begin{alltt}
tree-path (node)


   IN:node: nodo del que iniciar la búsqueda
   OUT: Una lista de los nombres de los planetas que forman el camino

\end{alltt}
\end{aibox}

\begin{aibox}{\examples}
\begin{alltt}
Examples 

(TREE-PATH *node-01*)

SIRTIS DAVION KATRIL KENTARES AVALON)


\end{alltt}
\end{aibox}

\begin{aibox}{\comments}
   Devuelve el camino a recorrer para llegar desde un nodo origen. En la lista devuelta, el primer elemento es el destino, y el último el origen.
   a un estado meta (por defecto también del problema *galaxy-m35*)


\end{aibox}

\begin{aibox}{\pseudocode}
\begin{alltt}
   Utilizando una función auxiliar de la forma:
   aux\_fun (nodo acumulador)
       si (nodo no es nulo)
           añadir el nodo al acumulador
           aux\_fun(padre(nodo) acumulador)
       sino
           devolver acumulador.

\end{alltt}
\end{aibox}

\begin{aibox}{\code}
\begin{alltt}
(defun get-states (node acc)
    (if (null node)
        acc
        (get-states (node-parent node) (append acc (list (node-state node))))))

(defun tree-path (node)
    (get-states (tree-search-aux *galaxy-M35* *A-star* (list node)) ()))





\end{alltt}
\end{aibox}
