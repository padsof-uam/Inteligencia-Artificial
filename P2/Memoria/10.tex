\begin{aibox}{\function}
\begin{alltt}
 action-sequence (node)


IN:  node: nodo origen desde el que empezar la búsqueda.
OUT: una lista de acciones que llevan desde el nodo origen hasta el destino.

\end{alltt}
\end{aibox}

\begin{aibox}{\examples}
\begin{alltt}
Examples

(ACTION-SEQUENCE *node-01*)

(\#S(ACTION :NAME NAVIGATE-WORM-HOLE :ORIGIN DAVION :FINAL SIRTIS :COST 8)
 \#S(ACTION :NAME NAVIGATE-WORM-HOLE :ORIGIN KATRIL :FINAL DAVION :COST 1)
 \#S(ACTION :NAME NAVIGATE-WHITE-HOLE :ORIGIN KENTARES :FINAL KATRIL :COST 2)
 \#S(ACTION :NAME NAVIGATE-WORM-HOLE :ORIGIN AVALON :FINAL KENTARES :COST 4))


\end{alltt}
\end{aibox}

\begin{aibox}{\comments}
Obtención de la secuencia de acciones de un camino, desde un nodo origen hasta un nodo meta
    (definido por *galaxy-M35*, según la estrategia *A-star*)


\end{aibox}

\begin{aibox}{\pseudocode}
\begin{alltt}
Utilizando una función auxiliar de la forma:
aux\_fun (nodo acumulador)
       si (nodo no es nulo)
           añadir acción(nodo) al acumulador
           aux\_fun(padre(nodo) acumulador)
       sino
           devolver acumulador

\end{alltt}
\end{aibox}

\begin{aibox}{\code}
\begin{alltt}
(defun  \_aux\_action-sequence (node acc)
    (if (null (node-action node))
        acc
        (\_aux\_action-sequence (node-parent node) (append acc (list (node-action node))))))

(defun  action-sequence (node)
    (\_aux\_action-sequence (tree-search-aux *galaxy-M35* *A-star* (list node)) ()))




\end{alltt}
\end{aibox}
