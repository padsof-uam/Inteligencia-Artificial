\begin{aibox}{\function}
\begin{alltt}
tree-search (problem strategy)

   IN: problem     estructura con la información del problema.
       strategy     estrategia a seguir para la búsqueda.
   OUT: Evalúa a un único nodo meta o nil si no hay solución.

\end{alltt}
\end{aibox}

\begin{aibox}{\examples}
\begin{alltt}
Examples
(tree-search *galaxy-M35* *A-STAR*)
\#S(NODE
   :STATE SIRTIS
   :PARENT \#S(NODE
              :STATE DAVION
              :PARENT \#S(NODE
                         :STATE KATRIL
                         :PARENT \#S(NODE
                                    :STATE KENTARES
                                    :PARENT NIL
                                    :ACTION NIL
                                    :DEPTH 0
                                    :G 0
                                    :H NIL
                                    :F 0)
                         :ACTION \#S(ACTION
                                    :NAME NAVIGATE-WHITE-HOLE
                                    :ORIGIN KENTARES
                                    :FINAL KATRIL
                                    :COST 2)
                         :DEPTH 1
                         :G 2
                         :H 3
                         :F 5)
              :ACTION \#S(ACTION
                         :NAME NAVIGATE-WORM-HOLE
                         :ORIGIN KATRIL
                         :FINAL DAVION
                         :COST 1)
              :DEPTH 2
              :G 3
              :H 1
              :F 4)
   :ACTION \#S(ACTION
              :NAME NAVIGATE-WHITE-HOLE
              :ORIGIN DAVION
              :FINAL SIRTIS
              :COST 1)
   :DEPTH 3
   :G 4
   :H 0
   :F 4)


\end{alltt}
\end{aibox}

\begin{aibox}{\comments}
Realiza la búsqueda en un problema según una estrategia

\end{aibox}

\begin{aibox}{\pseudocode}
\begin{alltt}
   Utilizamos una auxiliar recursiva con la lista de nodos.
   aux\_fun(problema estrategia lista\_nodos)
       si lista\_nodos no está vacía
           si ( es\_solución first lista\_nodos)
               first lista\_nodos
           sino
               nueva\_lista\_nodos = insertar
                         (lista\_nodos (expandir\_nodo first lista\_nodos) estrategia)
               aux\_fun (problema estrategia nueva\_lista\_nodos)
   

\end{alltt}
\end{aibox}

\begin{aibox}{\code}
\begin{alltt}
(defun tree-search-aux (problem strategy open-nodes)
    (when open-nodes
        (let ((n (first open-nodes)))
            (if (fncall (problem-fn-goal-test problem) (node-state n))
              n
              (tree-search-aux problem strategy  
                    (insert-nodes-strategy 
                        (expand-node n problem)
                        (rest open-nodes)
                        strategy))))))

(defun tree-search (problem strategy)
  (tree-search-aux problem strategy (list (make-node
                                          :state (problem-initial-state problem)
                                          :depth 0
                                          :g 0
                                          :f 0))))

\end{alltt}
\end{aibox}
